\documentclass{qutad} % For APA bibliography style
%\documentclass[ieee]{qutad} % For IEEE bibliography style

% Type Title in Title Case [Initial Caps]: Do Not Capitalize Coordinating Conjunctions [and, but, for, nor, or, so, yet], Prepositions, and Articles
\title{Title of The Thesis}

% Your name, use the official name on your transcript.
\author{Full Official Name}

% Your name in the format [LAST NAME, FIRST NAME, MIDDLE INITIAL.]
% Used in the abstract
\def \authorabstract{Name, Full, O.}

% Degree conferral date (January or June)
\def \submittedmonth{June}

% Degree conferral year
\def \submittedyear{2020}

% Year in which the copyright is secured by publication of the dissertation.
\def \copyrightyear{2020}

% Date of the defense
\def \defensedate{01/06/2020}

% The college
\def \college{College of Engineering}

% Name of the Dean of your college
\def \collegedean{Khalid Kamal Naji}

% Thesis for a master's, Dissertation for a PhD
\def \degreetype{Thesis}

% Check with your graduate coordinator for the title of your program's degree.
\def \degree{Master of Science in Computing}

% Full name of your adviser
\def \adviser{Dr. Supervisor}

% Co-adviser, if needed
%\def \coadviser{Dr. Coadvisor}

% Name of first committee member
\def \committeeone{Dr. Committee One}

% Name of second committee member
\def \committeetwo{Dr. Committee Two}

% Name of third committee member, if needed.
%\def \committeethree{Dr. Committee Three}

\abstract{
  An abstract is a concise account of the thesis or dissertation and should state the problem, describe the procedure or method used, and summarize the conclusions reached. An abstract is required for all papers. A maximum of 350 words are recommended for dissertations and a maximum of 150 for theses. Format the paragraphs with the same layout used in the document. All lines on this page are double spaced.
}

\def \dedication {
  A simple, optional note dedicating the work to a single person or small group of persons.

  The dedication is centered, typically in italic and rarely more than 3-4 lines.
}

\def \acknowledgments {
  The Acknowledgments page is optional. This page includes a brief, professional acknowledgment of the assistance received from individuals, advisor, faculty, and institution.
}

\begin{document}

\makefrontmatter

\chapter{Introduction}
Sample chapter with a sample figure \ref{thefig}.
\lipsum

\begin{figure} [H]
  \centering
  \includegraphics[width=0.7\linewidth]{samplefig.png}
  \caption{Sample figure}
  \label{thefig}
\end{figure}

\chapter{Another chapter}
Here are some sections and subsections. This is level 1.
\lipsum

\section{Sample section}
This is level 2.
\lipsum

\subsection{Sample subsection}
This is level 3.
\lipsum

\subsubsection{Sample subsubsection}
This is level 4. Table \ref{sampletable} is a sample table with a sample citation in it.
\lipsum

\begin{table}[H]
  \centering
  \setstretch{1}
  \caption{Sample table \cite{samplebib}}
  \label{sampletable}
  \begin{tabular}{x{3cm} x{3cm} x{3cm}}
    \hline
    A & B & C \\~\\
    \hline
    1 & 2 & 3 \\
    4 & 5 & 6 \\
    \hline
  \end{tabular}
\end{table}

\lipsum

{
  % add the Bibliography to the Table of Contents
  \cleardoublepage
  \ifdefined\phantomsection
  \phantomsection  % makes hyperref recognize this section properly for pdf link
  \else
  \fi
  \addcontentsline{toc}{chapter}{References}
}

\printbibliography[title = {REFERENCES}]

% Appendices if you need them. Remove this section if you don't.
\begin{appendices}
  % If you only have one appendix, uncomment the following line
  % to make sure it is just called "Appendix" and not "Appendix A".
  % \renewcommand*{\thechapter}{}

  \chapter{Some extra stuff} \label{app1}
  Sample appendix.
  \lipsum
  
  \chapter{Other extra stuff} \label{app2}
  Another appendix.
  \lipsum

\end{appendices}

\end{document}
