% Created 2020-08-12 Wed 01:04
% Intended LaTeX compiler: pdflatex
\documentclass[ieee]{qutad}
\usepackage[utf8]{inputenc}
\usepackage[T1]{fontenc}
\usepackage{graphicx}
\usepackage{grffile}
\usepackage{longtable}
\usepackage{wrapfig}
\usepackage{rotating}
\usepackage[normalem]{ulem}
\usepackage{amsmath}
\usepackage{textcomp}
\usepackage{amssymb}
\usepackage{capt-of}
\usepackage{hyperref}
\def \authorabstract{NAME, YOUR, F.}
\def \submittedmonth{June}
\def \submittedyear{2020}
\def \copyrightyear{2020}
\def \defensedate{01/06/2020}
\def \college{College of Engineering}
\def \collegedean{Khalid Kamal Naji}
\def \degreetype{Thesis}
\def \degree{Master of Science in Computing}
\def \adviser{Dr. Supervisor}
\def \committeeone{Dr. Committee One}
\def \committeetwo{Dr. Committee Two}
\abstract{
An abstract is a concise account of the thesis or dissertation and should state the problem, describe the procedure or method used, and summarize the conclusions reached. An abstract is required for all papers. A maximum of 350 words are recommended for dissertations and a maximum of 150 for theses. Format the paragraphs with the same layout used in the document. All lines on this page are double spaced.
}
\def \dedication {
A simple, optional note dedicating the work to a single person or small group of persons.
The dedication is centered, typically in italic and rarely more than 3-4 lines.
}
\def \acknowledgments {
The Acknowledgments page is optional. This page includes a brief, professional acknowledgment of the assistance received from individuals, advisor, faculty, and institution.
}
\author{Your Full Name}
\date{\today}
\title{Title of The Thesis}
\hypersetup{
 pdfauthor={Your Full Name},
 pdftitle={Title of The Thesis},
 pdfkeywords={},
 pdfsubject={},
 pdfcreator={Emacs 26.3 (Org mode 9.1.9)}, 
 pdflang={English}}
\begin{document}

\makefrontmatter

\chapter{Introduction}
\label{sec:orgeccc783}
Sample chapter with a sample figure \ref{thefig}.
\lipsum

\begin{figure}[htbp]
\centering
\includegraphics[width=.9\linewidth]{./samplefig.png}
\caption{Sample figure \label{thefig}}
\end{figure}

\chapter{Another chapter}
\label{sec:org3b8299d}
Here are some sections and subsections. This is level 1.
\lipsum

\section{Sample section}
\label{sec:org75cca43}
This is level 2.
\lipsum

\subsection{Sample subsection}
\label{sec:orgcbb5461}
This is level 3.
\lipsum

\begin{enumerate}
\item Sample subsubsection
\label{sec:org7f6d69d}
This is level 4. Table \ref{sampletable} is a sample table with a sample citation in it.
\lipsum

\begin{table}[htbp]
\caption{Sample table \cite{samplebib} \label{sampletable}}
\centering
\begin{tabular}{x{3cm} x{3cm} x{3cm}}
\hline
A & B & C\\
\hline
1 & 2 & 3\\
4 & 5 & 6\\
\hline
\end{tabular}
\end{table}

\lipsum

{
% add the Bibliography to the Table of Contents
\cleardoublepage
\ifdefined\phantomsection
\phantomsection  % makes hyperref recognize this section properly for pdf link
\else
\fi
\addcontentsline{toc}{chapter}{References}
}

\printbibliography[title = {REFERENCES}]

\begin{appendices}
% If you only have one appendix, uncomment the following line
% to make sure it is just called "Appendix" and not "Appendix A".
% \renewcommand*{\thechapter}{}
\chapter{Some extra stuff} \label{app1}
Sample appendix.
\lipsum
\chapter{Other extra stuff} \label{app2}
Another appendix.
\lipsum
\end{appendices}
\end{enumerate}
\end{document}
